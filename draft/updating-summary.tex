Let $G^- = \langle V^-, E^-, \textmd{cmd}^-, \overline{u}, \overline{r},\ell_s,\ell_e \rangle$ be a CFG obtained from Section~\ref{subsection:under-approximation} and $L_\fun{f}=\{(\ell_s^{i,\fun{f}},\ell_e^{i,\fun{f}}):\ell_s^{i,\fun{f}}\in E^- \wedge \ell_e^{i,\fun{f}}\in E^-\mbox{ for some }i\}$
the set of entry and exit locations of the call to the function $\mathtt{f}$.
%\mbox{ the pair of locations }\ell_s^{i,\fun{f}}\mbox{ and }\ell_e^{i,\fun{f}}\mbox{ in }E\mbox{ for some }i\}$ 
When the function UpdateSummary ($P$, $G^-$,$\Pi (G^-, \TT)$) is triggered, it extracts summaries from the inductive invariant $\Pi (G^-, \TT)= \{ I_\ell : \ell \in V^-
\}$ (Algorithm~\ref{algorithm:update-summary}). 

\begin{algorithm}
  \KwIn{$P=\{\mbox{CFG }G_\fun{main}\}\cup\{\mbox{CFG } G_\fun{f} : \fun{f}\mbox{ is a function}\}$: a program; $G^-$: a CFG after unwinding;$\Pi (G^-, \TT)= \{ I_\ell : \ell \in V^-
  \}$ : an inductive invariant of $G^-$}
  \KwOut{$S$:function summaries}

  \ForEach{function $\fun{f}$ with $G_\fun{f}\in P$}
  {	
  	Compute the set $L_\fun{f}$\;
  	$S[\fun{f}]:=\TT$\;
 	\ForEach{pair of locations $(\ell,\ell')\in L_\mathtt{f}$}
   	{
   		\lIf{$I_\ell$ does not contain any return variable of $\mathtt{f}$}
       	{
         		$S[\fun{f}]:=S[\fun{f}]\wedge \forall P_\mathtt{f}. (I_\ell \implies I_{\ell'})$\;
       	}	
   	}
    
  }
 
  \Return $S$\;
  \caption{$\textmd{UpdateSummary} (P, G^-,\Pi (G^-, \TT))$}
  \label{algorithm:update-summary}
\end{algorithm}

For each function $\fun{f}$ in the program $P$, we first compute the set $L_\fun{f}$  and initialize its summary $S[\fun{f}$ to $\TT$.
Let $V_\fun{f}$ be the set of all variables appearing in $G^-$ except the set of formal parameter and return variables of $\fun{f}$.
For each pairs of locations $(\ell,\ell')$ in the set $L_\fun{f}$, if the invariant of location $i$ does not contains any return variables of $\mathtt{f}$, we update $S[\fun{f}]$ to the formula $S[\fun{f}]\wedge \forall P_\mathtt{f}. (I_\ell \implies I_{\ell'})$.

\begin{figure}
  \centering
  \begin{tikzpicture}[->,>=stealth',shorten >=1pt,auto,node
      distance=2cm,thick,node/.style={circle,draw}]

      \draw [fill=gray!10] (3, -1) ellipse (1 and 1.5);
      \node (text) at (3, -1) {$\textsc{rename}(G_\fun{f})$};
      \node[node] (00) at (0, 0)  {$\ell$};
      \node[node] (01) at (0, -2) {$\ell'$};
      \node[node] (10) at (3, 0)  {\smallnode{$\ell^{i, \fun{f}}_s$}};
      \node[node] (11) at (3, -2) {\smallnode{$\ell^{i, \fun{f}}_e$}};

      \node (arrow_s) at (4.5, -1) {};
      \node (arrow_e) at (6.5, -1) {};
      
      \node[rectangle,text centered,draw] (text) at (9, -1)
      { add $\forall P_{\mathtt{f}}.
        (I_{\ell^{i, \mathtt{f}}_s} \implies I_{\ell^{i, \mathtt{f}}_e})$
        to $S[\mathtt{f}]$}; 
      
      \path
        (arrow_s) edge [dotted]
                  node {} (arrow_e)
        (00) edge 
             node {$\overline{\mathtt{u}} := \overline{a}$} (10)

        (11) edge 
             node {$\overline{\mathtt{x}} := \mathtt{\overline{ret}^f}$} (01) ;
    \end{tikzpicture}

  \caption{Updating Summary}
  \label{figure:updating-summary}
\end{figure}

\begin{proposition}
  \label{propposition:strengthen_postcondition}
  Let $G^- = \langle V^-, E^-, \textmd{cmd}^-, \overline{u}, \overline{r},\ell_s,\ell_e \rangle$ be a control flow graph obtained by unwinding $\fun{f}$ and $S=\textmd{UpdateSummary} (P, G^-,\Pi (G^-, \TT))$ be the summary computed by Algorithm~\ref{algorithm:update-summary}.
  If $\assert{\top}\ \mathtt{\overline{r}} :=
    \mathtt{f} (\overline{\mathtt{u}})\ \assert{S[\mathtt{f}]}$ for every
  function $\mathtt{f}$, then $\assert{\top}\ \mathtt{\overline{ret}^f} :=
  \mathtt{f} (\overline{\mathtt{u}})\ \assert{I_\ell \implies I_{\ell'}}$ for all $(\ell,\ell')$ in $L_\mathtt{f}$.
\end{proposition}
The above proposition hold because $S[\mathtt{f}]$ implies $\forall P_\mathtt{f}. (I_\ell \implies I_{\ell'})$, which implies $I_\ell \implies I_{\ell'}$.

\begin{proposition}
\label{propposition:invariant}
$\assert{I_\ell}\ \mathtt{\overline{r}} :=
    \mathtt{f} (\overline{\mathtt{u}})\ \assert{I_\ell}$
\end{proposition}