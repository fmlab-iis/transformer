We first define the rename function $\textsc{rename}(G^\fun{f},i)$. It returns a CFG $\langle V_i, E_i,\textmd{cmd}_i^\fun{f}, \overline{\texttt{u}}_i^\fun{f}, \overline{\texttt{r}}_i^\fun{f},s_i,e_i \rangle$ obtained by first replacing every return command $\mathtt{return}\ \overline{q}$ by assignments to return variables $\overline{\texttt{r}}^\fun{f} := \overline{q}$ and then renaming all variables and locations in $G^\fun{f}$ with the index value $i$. The function $\textsc{unwind}(G^\fun{f})$ returns a CFG $K^\fun{f}$ obtained by replacing all function call edges in $G^\fun{f}$ with the CFG of the called function after renaming. In order to help extracting summaries from the $K^\fun{f}$, $\textsc{unwind}(G^\fun{f})$ annotates in $K^\fun{f}$ the outermost pair of the entry and exit locations ${s_i}$ and ${e_i}$ of each unwound function $\fun{g}$ with an additional superscript $\fun{g}$, i.e., $s_i^\fun{g}$ and $e_i^\fun{g}$. The formal definition is given below.

Given a CFG $G^\fun{f}=\langle
V, E,\textmd{cmd}^\fun{f}, \overline{\texttt{u}}^\fun{f}, \overline{\texttt{r}}^\fun{f},s,e \rangle$.
We use $\hat{E} =\{e\in E: \textmd{cmd}^\fun{f} (e)= (\overline{\texttt{x}}:=\mathtt{g}(\overline{p}))\}$ to denote the set of function call edges in $E$ and define a function $\textsc{idx}(e)$ that maps a call edge $e$ to a unique index value.
The function $\textsc{mark}_\fun{f}(G^\fun{g})$ returns a CFG that is identical to $G^\fun{g}$, except that, for the case that no location with superscript $\fun{g}$ appears in $V$ (the locations of $G^\fun{f}$), it annotates the entry and exit locations, $s_k$ and $e_k$, of the returned CFG with superscript $\fun{g}$, i.e., $s_k^\fun{g}$ and $e_k^\fun{g}$. Note that, for each unwinding of function call, we mark only the outermost pair of its entry and exit locations.
Formally, $\textsc{unwind}(G^\fun{f}) = \langle V_u, E_u,\textmd{cmd}_u^\fun{f}, \overline{\texttt{u}}^\fun{f}, \overline{\texttt{r}}^\fun{f},s,e \rangle$ such that 
(1) $V_u = V\cup \bigcup\{V_i:(\ell, \ell')\in \hat{E} \wedge \textmd{cmd}^\fun{f}(\ell, \ell')=(\overline{\texttt{x}}:=\mathtt{g}(\overline{p}))\wedge \textsc{idx}(\ell, \ell')=i \wedge 
\textsc{mark}_\fun{f}(\textsc{rename}(G^\fun{g},i))=
\langle V_i, E_i,\textmd{cmd}^\fun{g}_i, \overline{\texttt{u}}^\fun{g}_i, \overline{\texttt{r}}^\fun{g}_i,s',e'\rangle \}$ 
(2) $E_u= E \setminus \hat{E} \cup \bigcup\{E_i\cup\{(\ell, s'),(e', \ell')\}: (\ell, \ell')\in \hat{E} \wedge 
\textmd{cmd}^\fun{f}(\ell, \ell')=(\overline{\texttt{x}}:=\mathtt{g}(\overline{p}))\wedge \textsc{idx}(\ell, \ell')=i \wedge 
\textsc{mark}_\fun{f}(\textsc{rename}(G^\fun{g},i))=
\langle V_i, E_i,\textmd{cmd}^\fun{g}_i, \overline{\texttt{u}}^\fun{g}_i, \overline{\texttt{r}}^\fun{g}_i,s',e'\rangle \}$ with $\textmd{cmd}_u^\fun{f} (\ell, s') = (\overline{\texttt{u}}_i^\fun{g}:=\overline{p})$ and $\textmd{cmd}_u^\fun{f} (e', \ell') = (\overline{\texttt{x}}:=\overline{\texttt{r}}^\fun{g}_i)$.


%$\textsc{unwind}(G) = \langle
%V', E' \rangle$ such that 
%(1) For all edges $(\ell, \ell')\in E$ with $\textmd{cmd} (\ell, \ell')=$

\begin{figure}[t]
  \centering
  \begin{tikzpicture}[->,>=stealth',shorten >=1pt,auto,node
      distance=2cm,thick,node/.style={circle,draw}]

      \node[node] (0) at (-4, 0)  {$\ell$}; %[label=above:$\mathtt{main()}$]
      \node[node] (1) at (-4, -2) {$\ell'$};

      \draw [fill=gray!10] (4, -1) ellipse (1.8 and 1.5);
      \node (text) at (4, -1) {$\textsc{mark}_\fun{f}(\textsc{rename}(G_\fun{g},i))$};
      \node[node] (00) at (0, 0)  {$\ell$};
      \node[node] (01) at (0, -2) {$\ell'$};
      \node[node] (10) at (4, 0)  {\smallnode{$s_i^\fun{g}$}};
      \node[node] (11) at (4, -2) {\smallnode{$e_i^\fun{g}$}};
      \node (arrow_s) at (-2.5, -1) {};
      \node (arrow_e) at (-0.5, -1) {};

      \path
        (arrow_s) edge [dotted]
                  node {} (arrow_e)
        (0) edge 
            node {$\overline{\mathtt{x}} := \mathtt{g} (\overline{p})$} (1)

        (00) edge 
             node {$\overline{\mathtt{u}}_i^\fun{g} := \overline{p}$} (10)

        (11) edge 
             node {$\overline{\mathtt{x}} := \mathtt{\overline{r}_i^\fun{g}}$} (01) ;
    \end{tikzpicture}

  \caption{Unwinding Function Calls}
  \label{figure:unwinding}
\end{figure}


\begin{proposition}
  Let $G^\fun{f}$ be a control flow graph. $P$ and $Q$ are logic formulae with
  free variables over program variables of $G^\fun{f}$. $\assert{P}\ G^\fun{f}\ 
  \assert{Q}$ if and only if $\assert{P}\ \textsc{unwind} (G^\fun{f})\ \assert{Q}$.
\end{proposition}
Essentially, $G^\fun{f}$ and $\textsc{unwind} (G^\fun{f})$ represent the same function $\fun{f}$. The only difference is that the latter has more program variables after unwinding, but this does not affect the states over program variables of $G^\fun{f}$ before and after the function.
