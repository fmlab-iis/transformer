% Extension: using program analyzer + bounded model checking

The number of unwindings is perhaps the most important factor in our
recursive analysis technique (Table~\ref{table:experiments}). We find
that \textsc{CPAChecker} performs poorly when many unwindings are
needed. We however do not enable the more efficient block encoding in
\textsc{CPAChecker} for the ease of implementation. One can improve
the performance of our algorithm with the efficient but complicated
block encoding. A bounded analyzer may also speed up the
verification of bounded properties. 

Our algorithm extracts function summaries from inductive invariants. 
There are certainly many heuristics to optimize the computation of
function summaries. For instance, some program analyzers return error
traces when properties fail. Particularly, a valuation of formal
parameters is obtained when \textmd{CheckSummary}
(Algorithm~\ref{algorithm:check-summary}) returns $\FF$. If the
valuation is not possible in the $\mathtt{main}$ function, one can use
its inductive invariant to refine function summaries. We in 
fact exploit error traces computed by \textsc{CPAChecker} in the
implementation. 

\noindent
\textbf{Related Works.}
In~\cite{LalR08,LalR09}, a reduction technique for checking context-bounded concurrent programs to sequential analysis is developed. Numerous intraprocedural analysis techniques have been developed over the years. Many tools are in fact freely available (see, for instance, \textsc{Blast}~\cite{BeyerHJM07}, \textsc{CPAChecker}~\cite{BeyerK11}, and \textsc{UFO}~\cite{AlbarghouthiLGC12}). Interprocedural analysis techniques are also available (see~\cite{RepsHS95,BallR01,CousotCFMMMR05,CuoqKKPSY12,coverity,polyspace} for a hopelessly incomplete list). Recently, recursive analysis attracts new attention. The Competition on Software Verification adds a new category for recursive programs in 2014~\cite{svcomp14}. Among the participants, \textsc{CBMC}~\cite{ClarkeKL04}, \textsc{Ultimate Automizer}~\cite{HeizmannCDEHLNSP13}, and \textsc{Ultimiate Kojak}~\cite{Kojak} are the top three tools for the recursive category. Our work is inspired by \textsc{Whale}~\cite{AlbarghouthiGC12}. Similar to \textsc{Whale}, we apply a Hoare logic proof rule for recursive calls. However, our technique works on control flow graphs and builds on an intraprocedural analysis tool. It is hence very lightweight and modular. Better intraprocedural analysis tools easily give better recursive analysis through our technique. \textsc{Whale}, on the other hand, analyzes by exploring abstract reachability graphs. Since \textsc{Whale} extends summary computation and covering relations for recursion, its implementation is more involved. Although \textsc{Whale} is able to analyze recursive program in theory, its implementation does not appear to support this feature.
