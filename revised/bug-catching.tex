
Let $G^\fun{f} = \langle V, E, \textmd{cmd}^\fun{f}, \overline{\texttt{u}}^\fun{f}, \overline{\texttt{r}}^\fun{f},s,e \rangle$ be a control flow
graph. The control flow graph $\underline{G}^\fun{f} = \langle V, E,
\underline{\textmd{cmd}}^\fun{f}, \overline{\texttt{u}}^\fun{f}, \overline{\texttt{r}}^\fun{f},s,e \rangle$ is obtained by replacing every
function call in $G$ with $\mathtt{assume\ false}$
(Figure~\ref{figure:under-approximation}). That is,
\begin{equation*}
  \underline{\textmd{cmd}}^\fun{f} (\ell, \ell') =
  \left\{
    \begin{array}{ll}
      \textmd{cmd}^\fun{f} (\ell, \ell') & 
      \textmd{if } \textmd{cmd}^\fun{f} (\ell, \ell') 
      \textmd{ does not contain function calls}\\
      \mathtt{assume\ false} &
      \textmd{otherwise}
    \end{array}
  \right.
\end{equation*}

\begin{figure}[t]
  \centering
  \begin{tikzpicture}[->,>=stealth',shorten >=1pt,auto,node
      distance=2cm,thick,node/.style={circle,draw}]

      \node[node] (0) at (-4, 0)  {$\ell$}; %[label=above:$\mathtt{main()}$]
      \node[node] (1) at (-4, -2) {$\ell'$};

      \node[node] (00) at (0, 0)  {$\ell$};
      \node[node] (01) at (0, -2) {$\ell'$};
      \node (arrow_s) at (-2.5, -1) {};
      \node (arrow_e) at (-0.5, -1) {};

      \path
        (arrow_s) edge [dotted]
                  node {} (arrow_e)
        (0) edge 
            node {$\overline{\mathtt{x}} := \mathtt{g} (\overline{p})$} (1)

        (00) edge 
             node {$\mathtt{assume\ false}$} (01);
    \end{tikzpicture}

  \caption{Under-approximation}
  \label{figure:under-approximation}
\end{figure}

\begin{proposition}
  Let $G^\fun{f}$ be a control flow graph. $P$ and $Q$ are logic formulae with
  free variables over program variables of $G^\fun{f}$. If $\assert{P}
  G^\fun{f} \assert{Q}$, then 
  $\assert{P} \underline{G}^\fun{f} \assert{Q}$.
\end{proposition}
The above holds because the computation of $\underline{G}^\fun{f}$ under-approximates the computation of $G^\fun{f}$. If all computation of $G^\fun{f}$ from a state satisfying $P$ always ends with a state satisfying $Q$, the same should also hold for the computation of $\underline{G}^\fun{f}$.
